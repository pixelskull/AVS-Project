\chapter{Motivation und Grundlagen}
Das Projekt wird im Rahmen des Moduls \enquote{Architektur verteilter Systeme} im Masterstudiengang Computer Science, Fachrichtung Software Engineering, durchgeführt.  Das Ziel ist die Implementierung einer verteilten Architektur. Die notwendige Hardware wird von der Hochschule zur Verfügung gestellt, welche detailliert in Kapitel \ref{technischeVoraussetzungen} beschrieben wird.
Die Entwicklung der Software ist Kernbestandteil dieses Projekts. 
\section{Ziel des Projekts}
Zu Beginn des Projektes wurde ein Problem gesucht, dass auf Basis einer einer verteilten Architektur gelöst oder berechnet werden kann. Das Projekt-Team legte sich fest, dass das Problem aus der Domäne der \emph{IT-Sicherheit} stammen soll. 
%Definition verteilte Architektur
Definition eines \emph{verteilten Systems}:
\begin{quotation}
	\textit{\enquote{Ein verteiltes System ist eine Ansammlung unabhängiger Computer, die den Benutzern wie ein einzelnes kohärentes System erscheinen.}} \citep{tanenbaum}
\end{quotation}


\section{Geplantes Vorgehen}

\section{Technische Voraussetzungen}
\label{technischeVoraussetzungen}
%Wie sieht Hardware aus? Wie Netz? Zeit. Programmiersprache usw.

\section{Grundlagen}
\label{grundlagen}




