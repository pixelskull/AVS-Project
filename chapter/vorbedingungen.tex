\chapter{Vorbedingungen}

\section{Theoretische Überlegungen}
Wir werden das PublishSubscribe-Paradigma benutzen. Die Slaves werden sich somit beim Master \enquote{subscriben}, um die zu berechnenden Hashes zu erhalten.

\subsection{Pseudocode}
\texttt{Master-Klasse:}
\begin{lstlisting}[basicstyle=\ttfamily,numbers=left,numberstyle=\footnotesize\ttfamily,backgroundcolor=\color{sourcegray}]
erzeuge aus PW(variable Stellen) HASH mit SHA1
vergiss das PW
erzeuge STACK mit moeglichen Passworten
	WAEHREND HASH_generiert !HASH
		Sende an WORKER naechsten Eintrag im STACK
		Warte auf Rueckmeldung durch WORKER
    		Setze Stackeintrag auf durch zurueckgegebene
    		ID auf generiert 
    		Vergleiche HASH_generiert mit HASH
	END WAEHREND
Gebe gefundenes PW und den zugehoerigen HASH aus
END
\end{lstlisting}
\clearpage

\texttt{Worker-Klasse:}
\begin{lstlisting}[basicstyle=\ttfamily,numbers=left,numberstyle=\footnotesize\ttfamily,backgroundcolor=\color{sourcegray}]
suche nach Master
registriere bei Master
	WAEHREND Master Online
    		Bekomme Eintrag im STACK und ID des Eintrags
	    	Berechne SHA1 HASH
    		Sende HASH_generiert, ID des Eintrags an 
    		Master
	END WAEHREND
END
\end{lstlisting}