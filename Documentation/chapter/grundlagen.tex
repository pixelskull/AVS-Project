\chapter{Grundlagen des verteilten Systems}
\label{Grundlagen}
Dieses Kapitel bietet Informationen zur verwendeten Hard- und Software. Damit werden die Umstände, unter denen das Projekt durchgeführt wird, verdeutlicht. Außerdem wird die Reproduzierbarkeit und damit der wissenschaftliche Anspruch an das Projekt realisiert. 

\section{Hardware}
Das verteilte System wird auf einem Mac-Cluster implementiert, das von der Hochschule zur Verfügung gestellt wird. Es kann auf 10 Rechner zugegriffen werden, die mit \emph{pip02} bis \emph{pip11} gekennzeichnet sind. Auf allen Rechnern ist (Stand 18.12.2015) aktuellste Version des Betriebssystems El Capitain und der Entwicklungsumgebung Xcode installiert. Details zu den Softwareversionen sind in Kapitel \ref{softwarebasis} zu finden. \\
In der folgenden Liste sind die Details zu den einzelnen Rechnern zu finden:

\begin{itemize}
	\item\textbf{pip02: Mac Pro (Anfang 2008)}
	\item[]Seriennummer: CK8250EUXYL
	\item[]Prozessor: 	2 x 2,8 GHz Quad-Core Intel Xeon 
	\item[]RAM: 2 GB 800 MHz DDR2 FB-DIMM
	\item[]Grafikkarte: 	NVIDIA GeForce 8800 GT (512MB)
	\item[]OS 10.11.2 El Capitain
	\item[]Xcode 7.2, Swift 2.1.1
	\item[] 
	\item\textbf{pip03: Mac Pro (Anfang 2008)}
	\item[]Seriennummer: CK8250EUXYL
	\item[]Prozessor: 	2 x 2,8 GHz Quad-Core Intel Xeon 
	\item[]RAM: 2 GB 800 MHz DDR2 FB-DIMM
	\item[]Grafikkarte: NVIDIA GeForce 8800 GT (512MB)
	\item[]OS 10.11.2 El Capitain
	\item[]Xcode 7.2, Swift 2.1.1
	\item[] 
	\item\textbf{pip04: Mac Pro (Anfang 2008) }
	\item[]Seriennummer: CK8250EUXYL
	\item[]Prozessor: 2 x 2,8 GHz Quad-Core Intel Xeon 
	\item[]RAM: 2 GB 800 MHz DDR2 FB-DIMM
	\item[]Grafikkarte: 	NVIDIA GeForce 8800 GT  (512MB)
	\item[]OS 10.11.2 El Capitain
	\item[]Xcode 7.2, Swift 2.1.1
	\item[]
	\item\textbf{pip05: Mac Pro (Anfang 2008)}
	\item[]Seriennummer: CK8250EUXYL
	\item[]Prozessor: 2 x 2,8 GHz Quad-Core Intel Xeon 
	\item[]RAM: 2 GB 800 MHz DDR2 FB-DIMM
	\item[]Grafikkarte: 	NVIDIA GeForce 8800 GT  (512MB)
	\item[] OS 10.11.2 El Capitain
	\item[] Xcode 7.2, Swift 2.1.1
	\item[]
	\item[] 
	\item\textbf{pip06: Mac Pro (Anfang 2009)}
	\item[] Seriennummer: CK92608B20H
	\item[]Prozessor: 2 x 2,26 GHz Quad-Core Intel Xeon 
	\item[]RAM: 6 GB 1066 MHz DDR3 ECC
	\item[]Grafikkarte: 	NVIDIA GeForce GT 120 (512MB)
	\item[] OS 10.11.2 El Capitain
	\item[] Xcode 7.2, Swift 2.1.1
	\item[]
	\item[] 
	\item\textbf{pip07: Mac Pro (Anfang 2009)}
	\item[] Seriennummer: CK92608B20H
	\item[]Prozessor: 2 x 2,26 GHz Quad-Core Intel Xeon 
	\item[]RAM: 6 GB 1066 MHz DDR3 ECC
	\item[]Grafikkarte: 	NVIDIA GeForce GT 120 (512MB)
	\item[] OS 10.11.2 El Capitain
	\item[] Xcode 7.2, Swift 2.1.1
	\item[] 
	\item pip08: Mac Pro (Anfang 2009)
	\item[] Seriennummer: CK92608B20H
	\item[] Prozessor: 2 x 2,26 GHz Quad-Core Intel Xeon 
	\item[] RAM: 6 GB 1066 MHz DDR3 ECC
	\item[] Grafikkarte: NVIDIA GeForce GT 120 (512MB)
	\item[] OS 10.11.2 El Capitain
	\item[] Xcode 7.2, Swift 2.1.1
	\item\textbf{pip09: Mac Pro (Anfang 2009)}
	\item[] Seriennummer: CK92608B20H
	\item[] Prozessor: 2 x 2,26 GHz Quad-Core Intel Xeon 
	\item[] RAM: 6 GB 1066 MHz DDR3 ECC
	\item[] Grafikkarte: NVIDIA GeForce GT 120 (512MB)
	\item[] OS 10.11.2 El Capitain
	\item[] Xcode 7.2, Swift 2.1.1
	\item[] 
	\item\textbf{pip10: Mac Pro (Anfang 2009)}
	\item[] Seriennummer: CK92608B20H
	\item[] Prozessor: 2 x 2,26 GHz Quad-Core Intel Xeon 
	\item[] RAM: 6 GB 1066 MHz DDR3 ECC
	\item[] Grafikkarte: NVIDIA GeForce GT 120 (512MB)
	\item[] OS 10.11.2 El Capitain
	\item[] Xcode 7.2, Swift 2.1.1
	\item[] 
	\item\textbf{pip11: Mac Pro (Anfang 2009)}
	\item[] Seriennummer: CK92608B20H
	\item[] Prozessor: 2 x 2,26 GHz Quad-Core Intel Xeon 
	\item[] RAM: 6 GB 1066 MHz DDR3 ECC
	\item[] Grafikkarte: NVIDIA GeForce GT 120 (512MB)
	\item[] OS 10.11.2 El Capitain
	\item[] Xcode 7.2, Swift 2.1.1
\end{itemize}

Zur Netzverbindung wird ein Switch des Herstellers \emph{Netgear} eingesetzt. Die Modellbezeichnung lautet \emph{Netgear GS116}. Der Switch hat 16 Ports und unterstützt bis zu 1000 Megabit/s (Gigabit-LAN).

\section{Software}
\label{softwarebasis}
Da Rechner des Herstellers Apple eingesetzt werden, sind die Programmiersprachen \emph{Objective C} oder \emph{Swift} effizient einsetzbar, da Apple diese vorrangig unterstützt. Das eingesetzte Betriebssystem Mac OS X 10.11.2 (El Capitain) und die native Entwicklungsumgebung Xcode 7.2 weisen eine hohe Kompatibilität zu den genannten Programmiersprachen auf. \\
Da die Programmiersprache \emph{Swift} seit Version 2.0 quelloffen angeboten wird\footnote{\url{https://github.com/apple/swift}} und zudem die aktuellere der beiden genannten Sprachen ist, möchte das Projektteam primär auf Swift zurückgreifen. Da aktuell der Einsatz von Objective C noch Bestandteil von Swift ist, werden beide genannten Programmiersprachen zum Einsatz kommen. \\
\subsection{Versionsverwaltung}
Zur Versionierung des Programmcodes und zum vereinfachtem dezentralen Entwickeln wird die Programmcode-Plattform \url{www.github.com} eingesetzt. Die auf dem Versionsverwaltungs-System \emph{Git} basierende Plattform ermöglicht ein flexibles und kollaboratives Arbeiten am Projekt sowie der Dokumentation. \\
\subsection{Frameworks}
Damit im Projekt weitere Frameworks mit wenig Aufwand eingesetzt werden können, hat das Projektteam sich entschieden \emph{Carthage}\footnote{\url{https://github.com/Carthage/Carthage}} einzusetzen. Carthage ist ein \enquote{einfacher, dezentraler Dependency-Manager} und wird quelloffen zur Verfügung gestellt. Durch die Auflösung von Abhängigkeiten, beispielsweise von bestimmten Frameworks, wird das asynchrone und dezentrale Entwickeln weiter optimiert. \\
Als alternativer Dependency-Manager hätte sich das Werkzeug \emph{CocoaPods}\footnote{\url{https://github.com/CocoaPods/CocoaPods}} angeboten. Einer der großen Unterschiede in der Arbeitsweise der beiden Werkzeuge liegt in der Verwaltung der Dependencies. Während CocoaPods auf eine zentrale Verwaltung setzt, werden die Abhängigkeiten bei Carthage dezentral verwaltet. Durch die dezentrale Verwaltung wird unser Primärziel, das effektive kollaborative Arbeiten, besser abgedeckt. Zudem wird Carthage nicht so tief in das Entwicklungsprojekt in der Entwicklungsumgebung Xcode verwurzelt, als es bei CocoaPods der Fall wäre. Dadurch entsteht eine weniger starke Abhängigkeit von dem Werkzeug. Aus den genannten Gründen entschied das Projektteam sich gegen die Verwendung von CocoaPods. \\
Zudem wird die Plattform \emph{Node.JS} benutzt, um einen Kommunikationsserver zur Verfügung zu stellen. Weitere Informationen dazu sind in Kapitel \ref{implementation} zu finden. 


%======================================================================
%======================================================================
% Eintraege ins Glossar
\nomenclature{BruteForce-Angriff}{Entschlüsselung eines unbekannten Passwortes, indem alle möglichen Zeichenkombinationen durchprobiert werden, bis das korrekte Passwort ermittelt ist.\vspace{4mm}}

\nomenclature{Dictionary-Angriff}{Effiziente Methode zum Entschlüsseln eines unbekannten Passwortes. Basis dazu ist ein Wörterbuch mit häufig benutzten Passwörtern, die nach einer festgelegten Strategie durchprobiert werden.\vspace{4mm}}
\nomenclature{Master}{Instanz innerhalb des verteilten Systems, welche für die Verteilung der anfallenden Tätigkeiten verantwortlich ist.\vspace{4mm}}

\nomenclature{Worker}{Instanz innerhalb des verteilten Systems, welche für die Bearbeitung der anfallenden Tätigkeiten verantwortlich ist.\vspace{4mm}}


\nomenclature{Provider}{Synonym für den \emph{Master}.\vspace{4mm}}




