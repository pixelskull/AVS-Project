\chapter{Einführung}
Das Projekt wird im Rahmen des Moduls \enquote{Architektur verteilter Systeme} im Masterstudiengang Computer Science, Fachrichtung Software Engineering, durchgeführt.  Das Ziel ist die Implementierung einer verteilten Architektur. Die notwendige Hardware wird von der Hochschule zur Verfügung gestellt, welche detailliert in Kapitel \ref{technischeVoraussetzungen} beschrieben wird.
Die Entwicklung der Software ist Kernbestandteil dieses Projekts. 

 \section{Vorgehen}
Um grundlegende Kenntnisse im Bereich der verteilten Systeme zu erhalten, soll das Projekt mit einer Literaturrecherche beginnen. Dabei soll eine Literaturauswahl getroffen werden, auf Basis derer genügend Fachkenntnisse gewonnen werden können. \\
Danach soll die Projektidee konkretisiert und anschließend definiert werden, welche in Abschnitt \ref{Projektidee} ausführlicher beschrieben wird.\\
Nach der Definition der Projektidee soll die Konzeption des geplanten verteilten Systems geschehen. Das entstehende Konzept soll die Grundlagen für die folgende Implementation bereitstellen. \\
Der Kern des Projektes wird nach der Konzeption die nachfolgende Implementation darstellen, bei der die theoretischen Überlegungen in die Praxis umgesetzt werden. \\
Zum Abschluss der Dokumentation sollen die durchgeführten Aktionen zusammengefasst werden. Darüber hinaus sollen diese kritisch reflektiert werden, um so eine bessere Einordnung des Gesamtprojekts zu gewährleisten. Abschließend sollen in einem Ausblick mögliche Zukunftsperspektiven und Ansatzpunkte für die Fortsetzung dieses Projekts gegeben werden. 

\section{Projektidee}
\label{Projektidee}
Zu Beginn des Projektes wurde ein Problem definiert, das auf Basis einer einer verteilten Architektur gelöst werden kann. Das Projektteam entschied sich für ein Problem aus dem Fachgebiet der IT-Sicherheit.\\
Die Wahl fiel auf die Implementation eines BruteForce-Angriffs. Konkret bedeutet dies, dass durch Ausprobieren aller möglichen Zeichenkombinationen versucht wird ein Passwort zu entschlüsseln. Durch die hohe Rechenleistung eines verteilten Systems bietet dieses eine ideale Grundlage für die Durchführung eines BruteForce-Angriffs.\\
 Das zu entschlüsselnde Passwort wird zu Beginn vom Benutzer eingetragen und in Form eines Hashes hinterlegt. Der Hash stellt die Zielbedingung für die geplante Anwendung dar. Nun soll die verteilte Architektur die möglichen Passworte bzw. deren Hashes berechnen. Sobald ein berechneter Hash mit dem Zielhash übereinstimmt, ist das vorgegebene Passwort entschlüsselt. Weitere Details dazu werden in Kapitel \ref{ideeBruteForce} erläutert. \\
 


\section{Allgemeine Anforderungen}
\label{allgemeineAnforderungen}
Das geplante Projekt soll außerdem den allgemeinen Ansprüchen an ein verteiltes System genügen, die im folgenden beschrieben werden. \\
Eine mögliche Definition eines \emph{verteilten Systems} lautet wie folgt:
\begin{quotation}
	\textit{\enquote{Ein verteiltes System ist eine Ansammlung unabhängiger Computer, die den Benutzern wie ein einzelnes kohärentes System erscheinen.}} \citep{tanenbaum}
\end{quotation}

Aus diesem Zitat lässt sich unter anderem entnehmen, dass bei einem verteilten System mehrere Rechner eingesetzt werden, welche zwar unabhängig voneinander arbeiten können, aber nun als kohärentes System eingesetzt werden. \\
Nach \citep{tanenbaum} verfolgt ein verteiltes System zudem folgende Ziele: 
\begin{itemize}
	\item \textbf{Ein verteiltes System sollte Ressourcen leicht verfügbar machen.}
	Die einfache Verfügbarkeit von Ressourcen innerhalb der Systemkomponenten ist nach Aussage des Autors das Hauptziel eines verteilten Systems. Als Ressource kann dabei alles angesehen werden, was sich innerhalb des verteilten Systems befindet. Dies kann beispielsweise eine Datei, ein Drucker, Speichergeräte oder ähnliches sein. 
	\item \textbf{Es sollte die Tatsache vernünftig verbergen, dass Ressourcen über ein Netzwerk verteilt sind.}
	Das System soll sich bei der Benutzung so anfühlen, als würde es nur auf einem einzigen Rechner arbeiten. Die Tatsache, dass verschiedene Komponenten in einem Netz verteilt sind, soll für den Anwender nicht bemerkbar sein.
	\item \textbf{Es sollte offen sein.}
	Das bedeutet, dass ein verteiltes System \enquote{seine Dienste so anbietet, dass diese die Syntax und Semantik der Dienste beschreiben}. Die Benutzung der Dienste soll durch formalisierte Beschreibung für alle Komponenten einfach und effizient durchführbar sein. In der Regel geschieht die Benutzung mit Hilfe von Schnittstellen. Benutzt man für die Spezifizierung der Schnittstellen beispielsweise die Schnittstellendefinitionssprache \emph{Interface Definition Language (IDL)}, ist eine standardisierte Benutzung der Dienste möglich-das System ist damit \emph{offen}.
	\item \textbf{Es sollte skalierbar sein.}
In der rezitierten Literatur werden drei Faktoren genannt, mit denen die Skalierbarkeit eines verteilten Systems angegeben werden kann. Dies ist zum einen die Größe des verteilten Systems. Können zum System einfach neue Benutzer und Geräte hinzugefügt werden, ist eine gute Größen-Skalierbarkeit gegeben. Als zweiter Faktor wird die geografische Größe genannt, also die mögliche Entfernung zwischen verschiedenen Ressourcen. Der dritte Faktor ist die administrative Skalierbarkeit. Die administrative Skalierbarkeit ist gegeben, wenn eine einfache Verwaltung von diversen Organisationen möglich ist. 
\end{itemize}
