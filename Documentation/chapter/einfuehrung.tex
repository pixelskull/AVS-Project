\chapter{Motivation und Grundlagen}
Das Projekt wird im Rahmen des Moduls \enquote{Architektur verteilter Systeme} im Masterstudiengang Computer Science, Fachrichtung Software Engineering, durchgeführt.  Das Ziel ist die Implementierung einer verteilten Architektur. Die notwendige Hardware wird von der Hochschule zur Verfügung gestellt, welche detailliert in Kapitel \ref{technischeVoraussetzungen} beschrieben wird.
Die Entwicklung der Software ist Kernbestandteil dieses Projekts. 
\section{Ziel des Projekts}
Zu Beginn des Projektes wurde ein Problem gesucht, dass auf Basis einer einer verteilten Architektur gelöst oder berechnet werden kann. Das Projekt-Team legte sich fest, dass das Problem aus der Domäne der \emph{IT-Sicherheit} stammen soll. Aufgrund des hohen Rechenaufwands entschied das Projektteam sich zu einem BruteForce-Angriff. Konkret bedeutet dies, dass durch Ausprobieren aller möglichen Kombinationen versucht wird ein Passwort zu entschlüsseln. Das zu entschlüsselnde Passwort wird zu Beginn eingegeben und in Form eines Hashes hinterlegt. Der Hash stellt die Zielbedingung für die geplante Anwendung dar. Nun soll die verteilte Architektur die möglichen Passworte bzw. deren Hashes berechnen. Sobald ein berechneter Hash mit dem Zielhash übereinstimmt, ist das vorgegebene Passwort entschlüsselt. Weitere Details dazu werden in Kapitel \ref{ideeBruteForce} erläutert. \\

Das geplante Projekt soll außerdem den allgemeinen Ansprüchen an ein verteiltes System genügen, die im folgenden beschrieben werden. \\
Eine mögliche Definition eines \emph{verteilten Systems} lautet wie folgt:
\begin{quotation}
	\textit{\enquote{Ein verteiltes System ist eine Ansammlung unabhängiger Computer, die den Benutzern wie ein einzelnes kohärentes System erscheinen.}} \citep{tanenbaum}
\end{quotation}

Aus diesem Zitat lässt sich unter anderem entnehmen, dass bei einem verteilten System mehrere Rechner eingesetzt werden, welche zwar unabhängig voneinander arbeiten können, aber nun als kohärentes System eingesetzt werden. \\
Nach \citep{tanenbaum} verfolgt ein verteiltes System zudem folgende Ziele: 
\begin{itemize}
	\item \textbf{Ein verteiltes System sollte Ressourcen leicht verfügbar machen.}
	Die einfache Verfügbarkeit von Ressourcen innerhalb ist nach Aussage des Autors das Hauptziel eines verteilten Systems. Als Ressource kann dabei alles angesehen werden, was sich innerhalb des verteilten Systems befindet. Dies kann beispielsweise eine Datei, ein Drucker, Speichergeräte oder ähnliches sein. 
	\item \textbf{Es sollte die Tatsache vernünftig verbergen, dass Ressourcen über ein Netzwerk verteilt sind.}
	Das System soll sich bei der Benutzung so anfühlen, als würde es nur auf einem einzigen Rechner arbeiten. Die Tatsache, dass verschiedene Komponenten in einem Netz verteilt sind, soll für den Anwender nicht bemerkbar sein.
	\item \textbf{Es sollte offen sein.}
	Das bedeutet, dass ein verteiltes System \enquote{seine Dienste so anbietet, dass diese die Syntax und Semantik der Dienste beschreiben}. Die Benutzung der Dienste soll durch formalisierte Beschreibung für alle Komponenten einfach und effizient durchführbar sein. In der Regel geschieht die Benutzung mit Hilfe von Schnittstellen. Benutzt man für die Spezifizierung der Schnittstellen beispielsweise die Schnittstellendefinitionssprache \emph{Interface Definition Language (IDL)}, ist eine standardisierte Benutzung der Dienste möglich-das System ist damit \emph{offen}.
	\item \textbf{Es sollte skalierbar sein.}
In der rezitierten Literatur werden drei Faktoren genannt, mit denen die Skalierbarkeit eines verteilten Systems angegeben werden kann. Dies ist zum einen die Größe des verteilten Systems. Können zum System einfach neue Benutzer und Geräte hinzugefügt werden, ist eine gute Größen-Skalierbarkeit gegeben. Als zweiter Faktor wird die geografische Größe genannt, also die mögliche Entfernung zwischen verschiedenen Ressourcen. Der dritte Faktor ist die administrative Skalierbarkeit. Die administrative Skalierbarkeit ist gegeben, wenn eine einfache Verwaltung von diversen Organisationen möglich ist. 
\end{itemize}

\section{Grundlagen}
\label{Voraussetzungen}
In diesem Abschnitt wird die soft- und hardwareseitigen Basis des Projekts vorgestellt. 
\subsection{Hardwarebasis}
Das verteilte System wird auf einem Mac-Cluster implementiert, das von der Hochschule zur Verfügung gestellt wird. Es kann auf 10 Rechner zugegriffen werden, die mit \emph{pip02} bis \emph{pip11} gekennzeichnet sind. Auf allen Rechnern ist (Stand 18.12.2016) aktuellste Version des Betriebssystems El Capitain und der Entwicklungsumgebung Xcode installiert. Details zu den Softwareversionen sind in Kapitel \ref{softwarebasis} zu finden. \\
In der folgenden Liste sind die Details zu den einzelnen Rechnern zu finden:

\begin{itemize}
	\item\textbf{pip02: Mac Pro (Anfang 2008)}
	\item[]Seriennummer: CK8250EUXYL
	\item[]Prozessor: 	2 x 2,8 GHz Quad-Core Intel Xeon 
	\item[]RAM: 2 GB 800 MHz DDR2 FB-DIMM
	\item[]Grafikkarte: 	NVIDIA GeForce 8800 GT (512MB)
	\item[]OS 10.11.2 El Capitain
	\item[]Xcode 7.2, Swift 2.1.1
	\item[] 
	\item\textbf{pip03: Mac Pro (Anfang 2008)}
	\item[]Seriennummer: CK8250EUXYL
	\item[]Prozessor: 	2 x 2,8 GHz Quad-Core Intel Xeon 
	\item[]RAM: 2 GB 800 MHz DDR2 FB-DIMM
	\item[]Grafikkarte: NVIDIA GeForce 8800 GT (512MB)
	\item[]OS 10.11.2 El Capitain
	\item[]Xcode 7.2, Swift 2.1.1
	\item[] 
	\item\textbf{pip04: Mac Pro (Anfang 2008) }
	\item[]Seriennummer: CK8250EUXYL
	\item[]Prozessor: 2 x 2,8 GHz Quad-Core Intel Xeon 
	\item[]RAM: 2 GB 800 MHz DDR2 FB-DIMM
	\item[]Grafikkarte: 	NVIDIA GeForce 8800 GT  (512MB)
	\item[]OS 10.11.2 El Capitain
	\item[]Xcode 7.2, Swift 2.1.1
	\item[]
	\item\textbf{pip05: Mac Pro (Anfang 2008)}
	\item[]Seriennummer: CK8250EUXYL
	\item[]Prozessor: 2 x 2,8 GHz Quad-Core Intel Xeon 
	\item[]RAM: 2 GB 800 MHz DDR2 FB-DIMM
	\item[]Grafikkarte: 	NVIDIA GeForce 8800 GT  (512MB)
	\item[] OS 10.11.2 El Capitain
	\item[] Xcode 7.2, Swift 2.1.1
	\item[]
	\item[] 
	\item\textbf{pip06: Mac Pro (Anfang 2009)}
	\item[] Seriennummer: CK92608B20H
	\item[]Prozessor: 2 x 2,26 GHz Quad-Core Intel Xeon 
	\item[]RAM: 6 GB 1066 MHz DDR3 ECC
	\item[]Grafikkarte: 	NVIDIA GeForce GT 120 (512MB)
	\item[] OS 10.11.2 El Capitain
	\item[] Xcode 7.2, Swift 2.1.1
	\item[]
	\item[] 
	\item\textbf{pip07: Mac Pro (Anfang 2009)}
	\item[] Seriennummer: CK92608B20H
	\item[]Prozessor: 2 x 2,26 GHz Quad-Core Intel Xeon 
	\item[]RAM: 6 GB 1066 MHz DDR3 ECC
	\item[]Grafikkarte: 	NVIDIA GeForce GT 120 (512MB)
	\item[] OS 10.11.2 El Capitain
	\item[] Xcode 7.2, Swift 2.1.1
	\item[] 
	\item pip08: Mac Pro (Anfang 2009)
	\item[] Seriennummer: CK92608B20H
	\item[] Prozessor: 2 x 2,26 GHz Quad-Core Intel Xeon 
	\item[] RAM: 6 GB 1066 MHz DDR3 ECC
	\item[] Grafikkarte: NVIDIA GeForce GT 120 (512MB)
	\item[] OS 10.11.2 El Capitain
	\item[] Xcode 7.2, Swift 2.1.1
	\item\textbf{pip09: Mac Pro (Anfang 2009)}
	\item[] Seriennummer: CK92608B20H
	\item[] Prozessor: 2 x 2,26 GHz Quad-Core Intel Xeon 
	\item[] RAM: 6 GB 1066 MHz DDR3 ECC
	\item[] Grafikkarte: NVIDIA GeForce GT 120 (512MB)
	\item[] OS 10.11.2 El Capitain
	\item[] Xcode 7.2, Swift 2.1.1
	\item[] 
	\item\textbf{pip10: Mac Pro (Anfang 2009)}
	\item[] Seriennummer: CK92608B20H
	\item[] Prozessor: 2 x 2,26 GHz Quad-Core Intel Xeon 
	\item[] RAM: 6 GB 1066 MHz DDR3 ECC
	\item[] Grafikkarte: NVIDIA GeForce GT 120 (512MB)
	\item[] OS 10.11.2 El Capitain
	\item[] Xcode 7.2, Swift 2.1.1
	\item[] 
	\item\textbf{pip11: Mac Pro (Anfang 2009)}
	\item[] Seriennummer: CK92608B20H
	\item[] Prozessor: 2 x 2,26 GHz Quad-Core Intel Xeon 
	\item[] RAM: 6 GB 1066 MHz DDR3 ECC
	\item[] Grafikkarte: NVIDIA GeForce GT 120 (512MB)
	\item[] OS 10.11.2 El Capitain
	\item[] Xcode 7.2, Swift 2.1.1
\end{itemize}

Zur Netzverbindung wird ein Switch des Herstellers \emph{Netgear} eingesetzt. Die Modellbezeichnung lautet \emph{Netgear GS116}. Der Switch hat 16 Ports und unterstützt bis zu 1000 Megabit/s (Gigabit-LAN).

\subsection{Softwarebasis}
\label{softwarebasis}
Da Rechner des Herstellers Apple eingesetzt werden, sind die Programmiersprachen \emph{Objective C} oder \emph{Swift} effizient einsetzbar, da Apple diese vorrangig unterstützt. Das eingesetzte Betriebssystem Mac OS X 10.11.2 (El Capitain) und die native Entwicklungsumgebung Xcode 7.2 weisen eine hohe Kompatibilität zu den genannten Programmiersprachen auf. \\
Da die Programmiersprache \emph{Swift} seit Version 2.0 quelloffen angeboten wird\footnote{\url{https://github.com/apple/swift}} und zudem die aktuellere der beiden genannten Sprachen ist, möchte das Projektteam primär auf Swift zurückgreifen. Da aktuell der Einsatz von Objective C noch Bestandteil von Swift ist, werden beide genannten Programmiersprachen zum Einsatz kommen. \\

Zur Versionierung des Programmcodes und zum vereinfachtem dezentralen Entwickeln wird die Programmcode-Plattform \url{www.github.com} eingesetzt. Die auf dem Versionsverwaltungs-System \emph{Git} basierende Plattform ermöglicht ein flexibles und kollaboratives Arbeiten am Projekt sowie der Dokumentation. \\

Damit im Projekt weitere Frameworks mit wenig Aufwand eingesetzt werden können, hat das Projektteam sich entschieden \emph{Carthage}\footnote{\url{https://github.com/Carthage/Carthage}} einzusetzen. Carthage ist ein \enquote{einfacher, dezentraler Dependency-Manager} und wird quelloffen zur Verfügung gestellt. Durch die Auflösung von Abhängigkeiten, beispielsweise von bestimmten Frameworks, wird das asynchrone und dezentrale Entwickeln weiter optimiert. \\
Als alternativer Dependency-Manager hätte sich das Werkzeug \emph{CocoaPods}\footnote{\url{https://github.com/CocoaPods/CocoaPods}} angeboten. Einer der großen Unterschiede in der Arbeitsweise der beiden Werkzeuge liegt in der Verwaltung der Dependencies. Während CocoaPods auf eine zentrale Verwaltung setzt, werden die Abhängigkeiten bei Carthage dezentral verwaltet. Durch die dezentrale Verwaltung wird unser Primärziel, das effektive kollaborative Arbeiten, besser abgedeckt. Zudem wird Carthage nicht so tief in das Entwicklungsprojekt in der Entwicklungsumgebung Xcode verwurzelt, als es bei CocoaPods der Fall wäre. Dadurch entsteht eine weniger starke Abhängigkeit von dem Werkzeug. Aus den genannten Gründen entschied das Projektteam sich gegen die Verwendung von CocoaPods. 



