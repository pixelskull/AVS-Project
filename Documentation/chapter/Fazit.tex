\chapter{Fazit und Ausblick}
Das abschließende Kapitel fasst das Projekt zusammen und gibt somit einen Überblick Über die durchgeführten Tätigkeiten.\\
 Außerdem wird ein Ausblick gegeben, der Ideen zu möglichen Fortsetzungen dieses Projekts bietet. \\
 Abschließend wird das durchgeführte Projekt kritisch reflektiert. Dadurch soll die Einordnung des Gesamtprojekts verbessert werden.

\section{Zusammenfassung des Projekts}
Vor der eigentlichen Durchführung des Projekts wurde eine Literaturrecherche durchgeführt. Die Recherche hatte zwei Ziele: Zum einen wurde Fachliteratur ermittelt, um notwendige Kenntnisse zu verteilten Systemen zu gewinnen. Zum anderen brachte die Recherche den Vorteil, dass der Prozess der Ideenfindung verbessert wurde. Bei der Literaturrecherche stellte sich heraus, dass das Buch \citep{tanenbaum} eines der Standardwerke im Bereich der verteilten Systeme darstellt. Aus diesem Grund wurde das buch primäre Wissensquelle für dieses Projekt ausgewählt. \\
Nach dem Studium von \citep{tanenbaum} wurden in Kapitel \ref{allgemeineAnforderungen} abgeleitet, die das zu implementierende verteilte System erfüllen soll. \\
Nachdem die allgemeinen Anforderungen formuliert worden sind, wurden die Hard- und Softwaregrundlagen in Kapitel \ref{Grundlagen} definiert. In diesem Kapitel wird beispielsweise die zur Verfügung stehende Hardware beschrieben. Damit wird sichergestellt, dass die Reproduzierbarkeit des Projekts gegeben ist. \\
In Kapitel \ref{Konzeption} wurden die theoretischen Überlegungen für die Implementation beschrieben. Dazu gehört die Planung der Architektur oder der Kommunikation. Bei der Konzeption wurde besonderer Wert darauf gelegt, dass die in Kapitel \ref{allgemeineAnforderungen} zusammengefassten Aspekte in der Implementation beachtet werden. \\
Danach werden in Kapitel \ref{implementation} Details zur Implementation beschrieben. Der Fokus lag dabei nicht auf einer vollständigen Beschreibung des Programmcodes, sondern auf Beschreibung einiger Merkmale der Implementation. Auf die vollständige Beschreibung wurde verzichtet, da diese den Rahmen dieser Dokumentation sprengen würde. Stattdessen wurde der Programmcode kommentiert. Dies hat den Vorteil, dass eine semantische Trennung zwischen Implementationsdetails und Programmcodebeschreibung geschaffen wird. Dadurch wird das Leseverständnis dieser Dokumentation weiter erhöht. 

\section{Ausblick}
\label{ausblick}
%neue Techniken aus Swift 3. Beispielsweise wird durch mallocs verursachtes Problem wahrscheinlich effektiver verhindert, welches während Implementation aufgetreten ist. Maschinenlernen nutzen, um häufig benutzte Passworte zu erkennen und damit Effizienz zu steigern. Automatisierte Auswertung der Angriffsperformance.

\section{Kritische Würdigung}
\label{kritik}
%Tests nur rudimentär. Mehr Fokus auf theoretisch möglichen Race Conditions, weil diese während Implementation aufgetreten sind. 
